\section{Studies on muon fake rates}
\label{sec:MuonFakes}
%\newcommand{\quote}[1]{``#1''}
%quick introduction (mostly from Run1 int note)
One of the most problematic backgrounds in this study is represented
by the charmless two-body $B$ decays referred to as $B \to hh^\prime$,
$h$ being a charged $K$ or $\pi$. This background is topologically
identical to, and peaks under, the signal. 
The only handle we can exploit is the muon identification
capability of the ATLAS detector. For these decays to feed into
our events, the charged $K$ or $\pi$ has to be misidentified as a
muon. \\
In the previous analysis~\cite{Alpigiani:1756291} a dedicated analysis was carried out in order to reduce the fake rate thus reducing the contamination
from these background events. The misidentification fraction was found to be respectively 0.00076 
and 0.00101 for negative and positive $K$, and 0.00044 and 0.00042 for
 negative and positive $\pi$. The average misidentification fraction was $0.00067 \pm 0.00001$, for a muon efficiency of $95\%$.\\
Some of the variables used in the BDT of the previous analysis are now used in the muon quality definition, 
therefore we decide to check the misidentification fraction for the different muon qualities.\\


The study of fake muons has been performed on two MC samples of signal
(\Bsmumu) and charmless two-body decays ($B \to hh^\prime$). %, with $h$, $h^\prime$=$K$ or $\pi$).
These samples have been produced with full GEANT simulation in order
to accurately describe the hadrons after they leave the Inner Detector. % the calorimeter response.
The preselection described in Section~\ref{sec:CandidatesPreselection}
is applied to the events entering this study (except for the muon quality requirement). A di-muon trigger request
 (mu6mu4) is applied to the \Bsmumu sample, while events containing preselected hadrons 
from $B\to hh^\prime$ that are misidentified as muons
are required to satisfy a single muon trigger ({\it{mu4}}). Once an event has 
passed the preselction, the two final state particles are considered separetely.\\
Table~\ref{table:misidentfraction} show the misidentification fraction and the muon efficiency for the different muon qualities available and for the previous analysis.
\begin{table}[h]
  \begin{center}
    \begin{tabular}{| c | c | c | c | c | c |}
      \hline
      &run1 presel + trigger match&run1 fake-BDT&loose muons&medium muons&tight muons\\ \hline
      total&0.00181&0.00067&0.00221&0.00221&0.00109\\ \hline
      total +&&&0.00229&0.00228&0.00114\\ \hline
      total -&&&0.00214&0.00213&0.00105\\ \hline
      total $\pi$&&0.0004&0.0018&0.0018&0.00101\\ \hline
      total K&&0.0009&0.00264&0.00263&0.00118\\ \hline
      $\pi^+$&0.00121&0.00042&0.00177&0.00177&0.00101\\ \hline
      $\pi^-$&0.00116&0.00044&0.00183&0.00182&0.00101\\ \hline
      $K^+$&0.00263&0.00101&0.00281&0.00281&0.00127\\ \hline
      $K^-$&0.00207&0.00076&0.00246&0.00245&0.00109\\ \hline
      &&&&&\\ \hline
      $\mu$ eff&&0.95&0.997&0.996&0.935\\ \hline
      $\mu^+$ eff&&&0.997&0.996&0.935\\ \hline
      $\mu^-$ eff&&&0.997&0.997&0.935\\ \hline
    \end{tabular}
    \caption{misidentification fraction and muon efficiency for the three 
      available muon qualities (loose, medium and tight) and, for comparison, 
      from the previous analysis~\cite{Alpigiani:1756291}, for different edmixtures of positive and negative $\pi$ and $K$.
      Misidentification fraction is calculated as the ratio of the number 
      of hadrons after all selections and the quality requirements divided 
      by the total number of hadrons that pass all selections. Muons 
      efficiency is defined as the number of muons that pass all selections and the quality requirements divided
      by the total number of muons that pass all selections.}
    \label{table:misidentfraction}  
  \end{center}
\end{table}
Tight muons have a comparable misidentification fraction and muon efficiency to the Run1 BDT; their usage would 
imply an increase of about $\sim \times 1.6$ in the number of fakes than using a Run1-like BDT, 
for about the same muon efficiency.\\
In order to understand the effect of the tight muons usage, we compared the effect of a Run1-like BDT and the tight muons in the analysis.\\
Knowing the available statistics before and after the fake-BDT application in previous 
analysis~\cite{Alpigiani:1756291} and the muon efficiency and fake-rate, we computed the 
Run1 background composition in terms of fake and real muons; applying the properties of tight muons we can estimate the effect 
tight muons would have had on the Run1 analysis.
Tables~\ref{table:bkgInRun1_BDTandTightMu} and ~\ref{table:sigInRun1_BDTandTightMu} show the amount 
of statistics for the different components of the likelihood used in the fit for the signal yield extraction 
and the estimation of the statistics using tight muons. The combinatorial background doesn't increase significantly, 
the SS-SV (ans semileptonic) background increases of a factor $\sim \times 1.1$ and the paking background increases 
of a factor $\sim \times 2.6$. Signal reduction due to lower muon efficiency is almost negligibe.\\
\begin{table}[h]
  \begin{center}
    \begin{tabular}{| c | c | c | c | c |}
      \hline
       & comb bkg run1 & comb bkg tight mu & SS-SV bkg run 1 & SS-SV bkg tight mu\\  \hline
      bin 1 & 1455.3 & 1460.7 & 205.5 & 229.0\\  \hline
      bin 2 & 110.5 & 110.9 & 105.6 & 117.7\\  \hline
      bin 3 & 11.6 & 11.6 & 51.2 & 57.1\\  \hline
   \end{tabular}
    \caption{breakdown of the background contributions in the Run1 likelihood after the BDT application. 
    Columns labelled tight mu show the estimated statistcs we would have gotten using tight muons instead of the BDT.}
    \label{table:bkgInRun1_BDTandTightMu}
  \end{center}
\end{table}
\begin{table}[h]
  \begin{center}
    \begin{tabular}{| c | c | c |}
      \hline
       & run 1 & run 1 with tight mu  \\ \hline
      nBs (SM expected) & 41 & 39.7 \\ \hline
      nBd (SM expected) & 5 & 4.8 \\ \hline
      peaking bkg & 1 & 2.6 \\ \hline
    \end{tabular}
    \caption{Breakdown of the expected signal and peaking background contributions in the Run1 likelaihood after the BDT application. 
    Columns labelled tight mu show the estimated statistcs we would have gotten using tight muons instead of the BDT.}
    \label{table:sigInRun1_BDTandTightMu}
  \end{center}
\end{table}
In order to asses the impact of tight muons usage on the analysis performance a set of toy-MC 
based on the Run1 analysis likelihood with the available statistics modified according 
to the estimation of the available stastics in this analysis performed in section~\ref{sec:Trigger} 
has been run, varying the number of events associated to the different background and signal models according to the usage of tight muons 
or a Run1-like BDT. Table~\ref{table:tightMuAndBDTRMS} shows the root mean square of the distribution 
of the fitted number of $B_s$ and $B_d$ events. No significant effect on analysis sensitivity is 
visible for $B_s$, while a $\sim 1\%$ broader RMS is visible on $B_d$ count.\\
Given the negligible difference due to the usage of tight muons instead of a Run1-like BDT for fake 
muons reduction, we decide to use tight muons instead of developing and ad-hoc BDT for this analysis.
\begin{table}[h]
  \begin{center}
    \begin{tabular}{| c | c | c || c | c |}
      \hline
      &Run1 fake-BDT&Run1 tight muons&Run2 fake-BDT&Run2 tight muons\\ \hline
      RMS nBs&15.20 $\pm$ 0.01&15.16 $\pm$ 0.01&21.260 $\pm$ 0.015&21.260 $\pm$ 0.015\\ \hline
      RMS nBd&13.88 $\pm$ 0.01&13.96 $\pm$ 0.01&18.850 $\pm$ 0.015&19.130 $\pm$ 0.015\\ \hline
    \end{tabular}
    \caption{RMS of distributions of number of fitted $B_s$ and $B_d$ events in toy-MC study. The columns labelled Run1 
      refer to the statistcs available in the previous analysis, modified for the usage of tight muons in case of the 
    second column. Columns labelled Run2 refer to the estimation of the statistics available for this analysis performed in section~\ref{sec:Trigger}.}
    \label{table:tightMuAndBDTRMS}
  \end{center}
\end{table}
\clearpage
