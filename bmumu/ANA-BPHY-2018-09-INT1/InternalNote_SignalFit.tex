\section{SignalFit}
\label{sec:SignalFit}

To extract the signal yield, an unbinned maximum-likelihood fit is performed on
the selected events, mostly based on the fit performed in the previous analysis, 
based on the full Run1 dataset. The fit is performed on the invariant mass distribution,
classifying the events according to different intervals in the continuum-BDT
output. This is similar to the strategy used by CMS and LHCb.\\
The events have been classified according to the \yel[still work in progress]{three} bins in
continuum BDT$_{ss}$ output. The bins are chosen to correspond
to a signal efficiency equal to \yel[still work in progress]{$.....~\%$}, and they result ordered
according to increasing signal-to-noise ratio. \\
The sensitivity of our study is discussed in this chapter. The model for
describing signal and background is based on MC and on data collected in
the sidebands of the search region. 
\begin{itemize}
    \setlength{\itemsep}{0pt}%
    \setlength{\parskip}{0pt}%
\item The models used for the signal and the backgrounds are discussed in
Sections~\ref{sub:sec:signal-description}%, \ref{sub:sec:bkg-description-1}
and~\ref{sub:sec:bkg-description-2}.
\item The results of the fit to the data in the sidebands and the interpolation
in the signal region are presented in Section~\ref{sub:sec:bkg-normalisation-fit}.
\item A summary of the baseline fit configuration is given in Section~\ref{sub:sec:fit_summary}.
%\item The studies of the sensitivity to the signal, obtained with toy experiments,
%are discussed in Sections~\ref{sub:sec:bkg-signal-fit-1} and~\ref{sub:sec:bkg-signal-fit-2}.
\item Systematic uncertainties on the fit are discussed in Section~\ref{sub:sec:fit_syst}.
\end{itemize}

\subsection{Signal and peaking background}
\label{sub:sec:signal-description}
The mass shape of the {\bf {\boldmath $\Bsmumu$}}, as well as the one of {\bf {\boldmath $\Bdmumu$}},
is described by a superposition of two Gaussian distributions, 
both centred at the world average value of the mass.
The parameters of this distribution will be extracted from MC. % with the fitsshown in figure~\ref{fig:Bs-mass-fit}. 
%The mass PDF is found to be independent from the output of the continuum BDT used to reduce the combinatorial background.


The {\bf peaking background} is composed of \Bhh, mainly $B_s \to K^+K^-$ and
$B_d \to K^\pm\pi^\mp$, in which both hadrons are misidentified as muons. 
Due to the mass distortion related to the $K \to \mu$ mass assignment, and the
smaller one for $\pi \to \mu$, the mass distribution of these events is substantially
superimposed with the $B_d$ signal.


%\subsection{Background components}
%\label{sub:sec:bkg-description-1}





\subsection{Parametrisation of background components}
\label{sub:sec:bkg-description-2}
The {\bf combinatorial\/} (opposite-side) background, 
following the Run1 analysis, will be  described with a Chebychev
first order polynomial like $f(x) = 1 + \alpha \,T_1(x) = 1 +\alpha\, x / 1200 MeV$.\\
The  {\bf same-side and same-vertex (SS+SV)} background includes double
semileptonic cascade events (e.g., $B \to D\mu X \to \mu \mu X^\prime$),
which we call SS, where the muons do not originate from the same vertex,
and events where the muons come from the same vertex (e.g., $B \to K\mu\mu$),
which we call SV. 
In both cases, in Run1 the mass distribution of the two muons was peaked far below
the signal region, and the analysis was sensitive to a tail of the distribution
determined by kinematic limits and detector resolution effects. 
In Run1 this background was fitted with an exponential PDF $f(x) = \exp(\alpha x)$
used for MC events of this class, we expect to ba able to use the same model. \\
%The {\boldmath \Bc} component of the background is mainly due to a small number
%of events in which \Bc\ decays into $J\!/\!\psi\mu\nu$ feed into our sample.
%Their BDT values are distributed between the signal-like and
%background-like values. The mass shape is smoothly decreasing
%towards the signal region. It may be fitted with an exponential ($f(x) = \exp(\alpha x)$). 
%Figure~\ref{fig:Bcbins} shows the fit to the \Bc\ events in the invariant mass
%distribution in each BDT$_{ss}$ bin, superimposed with an exponential
%PDF parametrisation. Given the relatively smaller amplitude, these events
%are expected to feed into the continuum and partially in the SS+SV events.
%Therefore no PDF is added in the fit to model the \Bc\ component of the
%background as further discussed below. 


The {\bf semileptonic} background is due to few-body semileptonic 
$B$ decays feeding into our final selections though a misidentification
$h \to \mu$, in the limit of low energy neutrinos.
In particular  $B_d \to \pi\mu\nu$ and $B_s \to K\mu\nu$ can contribute,
together with $\Lambda_b \to p\mu\nu$. The mass distribution for the last
process extends closer to the signal region, but it is highly suppressed
because of a very low probability of misidentifying the proton as muon in
ATLAS.
The contribution from this  background is expected to be significantly smaller than the SS+SV
and the combinatorial background contributions in all bins of the BDT output. 
It is expected to be described with 
sufficient accuracy by the first-order polynomial and the exponential PDF used for the main 
background components, without adding an extra PDF.\\

\subsection{Fit to background components from MC and to sideband data}
\label{sub:sec:bkg-normalisation-fit}
In each bin in the continuum BDT, the background will been fitted in the
sideband of the data sample,  and interpolated in the search region. 
This is done in order to optimise the analysis and evaluate its sensitivity 
before proceeding to the unblinding of the signal region and performing a
simultaneous fit to signal and background.\\




\subsection{Summary of the fit configuration}\label{sub:sec:fit_summary}
The baseline signal fit to the number of events is expected to include the following PDFs: 
\begin{enumerate}
\item{signal PDF:} the mass dependence is described  by the sum of 2 Gaussians 
centred at the $B_s$ (or $B_d$) mass. The widths of the Gaussians and their relative fraction, assumed to
be identical in all continuum-BDT bins, will be taken from MC and fixed in the fit. 
%The BDT bins are defined to contain each 1/3 of the total number of signal events.
%Systematic uncertainties in these fractions are derived from the studies discussed in  Section~\ref{sec:BDTbineff},
%and are described with Gaussian constraints, so that  $33.3\pm4.3$\% of the signal events are 
%expected in bin-3, $33.3\pm2.0$\% are expected in bin-2, and the fraction in bin-1 is constrained so that the total 
%is equal to 100\%.\footnote{
%The uncertainty in the total efficiency of the continuum-BDT selection, summed over the three bins, 
%is included in the uncertainty of the efficiency ratio between \Bp\ and \Bds, discussed in Section~\ref{subsec:eff}}

\item{Continuum background PDF:} the mass dependence is first order polynomial. The normalisation and the 
slope will be extracted independently in each bin of the continuum-BDT.  
In full Run1 analysis Gaussian constraints were placed on the uniformity of the slope, so 
that the slope in bin-2  (bin-3) was equal to the one in bin-1 within $\pm 40$\% 
($\pm 80$\%). 

\item{Low-mass background PDF:} exponential dependence on the mass.
The normalisation will be extracted independently in each bin of the continuum-BDT, 
while the shape will be assumed to be uniform.

\item {Peaking background:} the mass dependence is described with a Gaussian describing the total
background. % shown in Figure~\ref{fig:massBhh}. 
%A Gaussian constraint of $1.0\pm0.8$ is applied to the normalisation. 
%The uncertainty is conservatively taken from the upper limit to the number of events containing pairs of 
%{\em fake muons}, studied through inversion of the corresponding selection cuts as discussed in Section~\ref{app:fake-sel-inverted} of the Appendix. This background is equally shared in the three bins of continuum-BDT. 
\end{enumerate}

%The following free parameters are determined by the fit:
%\begin{itemize}
%\item the total number of \Bs\ events, and the total number of \Bz\ events in case of simultaneous fit to both channels;
%\item the number of continuum background events in each continuum-BDT bin (3 free parameters);  
%\item the number of low-mass background events in each continuum-BDT bin (3 free parameters); 
%\item the parameter describing the shape of the continuum background in bin-1 of the continuum-BDT;
%\item the parameter describing the shape of the low-mass background.
%\end{itemize}

%The following parameters subject to Gaussian constraints are also determined by the fit:
%\begin{itemize}
%\item the fractions of signal events contained in each bin of the continuum-BDT;
%\item the parameters describing the shape of the continuum background in bin-2 and bin-3 of the continuum-BDT;
%\item the total number of peaking-background events;
%\item in case of simultaneous fit to the \Bs\ and \Bz components, two smearing parameters describing the correlated systematic error in the fit to the number of events.
%\end{itemize}

%The use of three bins in the BDT output is found to optimise the determination
%of the mass dependence of the two backgrounds, using events contained in the
%first and second bin, while the second and in particular the third bin provide
%sensitivity to the signal. 


\subsection{Systematic uncertainties on the fit in the simultaneous fit to $B_s$ and $B_d$}
\label{sub:sec:fit_syst}
The evaluation of the systematic uncertainties due to the fitting procedure 
will be evaluated as in the full Run1 analysis, by applying 
variations to the baseline model and testing the result beahviour 
with toy-experiments. The corresponding variations in the result of the fit in the baseline
configuration will be taken as systematic uncertainties.


\clearpage
