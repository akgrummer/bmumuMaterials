\section{Trigger}
\label{sec:Trigger}

In 2015-16 the main low \pt di-muon triggers were:
\begin{itemize}
\item {\it 2mu4} prescaled (2015) or not active (2016)
\item {\it mu6\_2mu4} prescaled during 2016, $\sim 82\%$ efficient w.r.t. 2mu4
\item {\it 2mu6} unprescaled, $\sim 35\%$ efficient w.r.t. 2mu4
\end{itemize}
The collected luminosity of each trigger per year are shown in table~\ref{table:collectedLumi}.
\begin{table}[h]
  \begin{center}
    \begin{tabular}{| c | c | c |}
      \hline
          &   2015  &  2016 \\  \hline
      2mu4 &  3.17 $\ifb$  & 0 $\ifb$ \\ \hline
      mu6mu4 & 3.93  $\ifb$  & 37.13 $\ifb$ \\ \hline
      2mu6 &  3.93  $\ifb$ & 26.03 $\ifb$ \\ \hline
   \end{tabular}
    \caption{Collected luminosity in 2015 and 2016.}
    \label{table:collectedLumi}
  \end{center}
\end{table}
%decide which triggers we want, possibilities
%2mu4+mu6mu4+2mu6  
%mu6mu4+2mu6
%mu6mu4 only GOOD

%some numbers
%total collected lumi     2015: 3.93 (2mu4 3.17)       2016:  37.13         mu6mu4 2016: 26.03
%2mu6+mu6mu4 = 28.5
%mu6mu4 = 24.6
%2mu4 addition would increase the 2mu6+mu6mu4 of a factor $\sim1%$

We can compare the available collected luminosity provided by the different triggers by calculating their effective collected luminosity, the total luminosity we would need to have the same amount of statistcs using 2mu4 as main trigger.
\begin{itemize}
\item consider mu6mu4 only would get an effective collected luminosity of 24.6 $\ifb$
\item consider the combination of mu6mu4 and 2mu6 would get an effective collected luminosity of 28.5 $\ifb$
\item 2mu4 addition would increase the 2mu6+mu6mu4 statistics of a factor $\sim 1 \%$
\end{itemize}
The prescaled trigger mu6mu4 provides the bulk of the statistics, which is $\sim 85 \%$ of total. The addition of 2mu6 would allow to use basically the full dataset but would also increase the complexity of the analysis, given the tight schedule, this round of the analysis will use mu6mu4 only.
%after that, talk about HLT-> _lxy0 and delayed stuff
%I have to say that L1_MU6_MU4 is always the same, but at HLT it changes, for 2015 HLT_mu6_mu4_bBmumu, for 2016 HLT_mu6_mu4_bBmumu_Lxy0 until some points and then HLT_mu6_mu4_bBmumu_Lxy0_delayed
%for the B+ don't really know
\add{still have to add the description of the different HLT algorithms (Lxy0, dalayed)}\\


In order compare the available statistics for this analysis with respect to the previous version, 
three main ingredients are needed.
\begin{enumerate}
\item \textbf{B production cross section with respect to Run-1}, due to the Run-2 
increased center of mass energy a value $\sim 1.7$ times higher is expected, 
according to studies performed using \yel[to be added in file bib]{FONLL ....}. %~\cite{FONLL}}.
\item \textbf{The efficiency of the dimuon triggers available in Run-2 with respect to the Run-1 triggers},  
already shown in this section.
\item \textbf{The collected luminosity for the selected trigger in Run-2}, an integrated luminosity 
of $\approx 40\ifb$ has been collected during the 2015 and 2016 data-taking period.
\end{enumerate}
The same signal over background ratio as the analysis performed on the Run-1 dataset is assumed for this study.\\
With the three ingredients listed above and the previous assumptions we estimate the expected 
statistics available for the 2015-16 dataset by scaling the Run-1 analysis 
statistics. This yields an estimated 2-fold increase with respect to Run-1 statistics.




\clearpage
