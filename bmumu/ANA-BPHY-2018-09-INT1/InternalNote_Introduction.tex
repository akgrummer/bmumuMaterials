\section{Introduction}
\label{sec:Introduction}

This document is the supporting note for the \Bmumu\ analysis
of the 2015+2016 part of the Run 2 dataset.
The aim is to obtain a first intermediate Run 2 ATLAS result on the $B_{s} \to \mu \mu$ and $B \to \mu \mu$ final states.

The strategy for this updated analysis is mostly following what adopted for the
complete run 1 version, focusing on the possibility of a measurement
of the $B_{s} \to \mu \mu$ branching fraction and taking advantage of the Run 2 statistics
available in this first Run 2 iteration.
Wherever possible the approach will be simplified in favor of a leaner analysis,
leaving -- within reason -- the ultimate exploitation of the sample sensitivity to for the full Run 2 dataset study.
The trigger and muon quality selections -- for instance -- are simplified with little loss in statistical power
and background rejection, but with a much simpler analysis strategy.

Theoretical prediction on the $\Bmumu$ branching ratios are
$\BR(\Bsmumu) = (3.65 \pm 0.23) \times 10^{-9}$ and
$\BR(\Bdmumu) = (1.06 \pm 0.09) \times 10^{-10}$~\cite{Bobeth:2013uxa}.
CMS and LHCb have a combined Run 1 result \cite{Aaij:2017vad,CMS:2014xfa} showing a >5$\sigma$ effect for the $B_{s} \to \mu \mu$ final state giving
an average branching ratio of $(2.8^{+0.7}_{-0.6}) \times 10^{-9}$,
and >3$\sigma$ evidence for $\Bdmumu$ with a central BR value of $3.6^{+1.6}_{-1.4} \times 10^{-10}$.
ATLAS has limited trigger efficiency and mass resolution, resulting
in a degraded sensitivity to these decays.~\footnote{We performed
  tests showing the detailed breakdown of the mass resolution and signal statistics contributions
  to the analysis, reported later in section \ref{sec:BranchingRatio}.}
The combined Run 1 ATLAS result shows a sensitivity comparable to expectations,
with a measured $\Bsmumu$ BR of $0.9^{+1.1}_{-0.8} \times 10^{-9}$ and an upper bound
on $\Bdmumu$ of $4.2 \times 10^{-10}$ at $95\%$ CL.
In order to exploit at best ATLAS data, we updated the analysis re-optimising the background
rejection and the BR extraction methodology.

The main idea guiding this analysis is to increase the sensitivity
to the signal by using a mass fit on the widest possible
set of events, and improve the $\Bdmumu/\Bsmumu$ correlation with
a better exploitation of the signal subsample with the smallest mass resolution.
 A loose selection will be applied to retain
a maximum of signal events, and the final fit will distinguish
between signal and the various backgrounds, as well as different resolution components
within the signal sample.
Finally the peaking background will be a major contributor under
the signal peaks, differing from dimuonic decays only through muon identification.
The improved Run 2 muon reconstruction allows this iteration of the analysis to rely
on standard MCP categories for the separation of $\Bhh$ from $\Bmumu$.

In this note, we refer mainly to the internal documentation on the previous
analysis on the full Run 1 dataset~\cite{Alpigiani:1756291,Aaboud:2016ire},
on the full 2011 data set~\cite{bsmumuv2} and the studies carried out for the first ATLAS analysis
in this mode~\cite{Aad:2012pn}. %and contained in the relative internal note \cite{bsmumuv1}.

The reference formula for the branching ratio measurement is similar to the
one previously used in~\cite{Aad:2012pn}:

\begin{eqnarray}
  \hbox{\hspace{-1cm}} \BR (\Bmumu)=&
  \hbox{\hspace{-3mm}}\BR (\BpmKpmJpsi\! \to \mumu K^\pm)\times
  \frac{f_{u}}{f_{s}}\times N_{\mumu}\times \left( N_{J/\psi K^\pm} \theInvRhok \right)^{-1}\, ,\label{eq:BRFormula}
\end{eqnarray}

and simplified with respect to the final Run 1 analysis thanks to the fact that we employ one single
trigger category encompassing $\sim 80-85\%$ of the signal for the 2015+2016 Run 2 dataset.\\
While the branching ratio $BR (\BpmKpmJpsi\! \to \mumu K^\pm)$ and the relative $B_u/B_d$ production
fraction $\frac{f_{u}}{f_{s}}$ are derived from other experimental results, a good fraction of this
document is devoted to the derivation of the remaining ingredients to this formula and their uncertainties.\\
The measured $\BpKpJpsi$ yield $N_{J/\psi K^\pm}$ will be derived in section \ref{sec:BPlusYield}, while
the relative efficiencies and acceptances of $\Bmumu$ and $\BpKpJpsi$ will be extracted in \ref{sec:EfficiencyAndAcceptance}.
The $\Bmumu$ yield will be derived from invariant mass distribution fits in section \ref{sec:SignalFit}, implementing
the selections optimized in sections \ref{sec:MuonFakes}, \ref{sec:ContinuumBDT} and \ref{sec:CutFlow}.\\
Given these ingredients, the branching ratio to the left of the reference formula above will be derived in
section \ref{sec:BranchingRatio}, where relevant systematic effects will be propagated as well.
Dedicated sections will discuss the tuning of Montecarlo samples \ref{sec:MCTuning}, the data-driven \ref{sec:DataMCComparison} extrapolation of
MC models to the signal region \ref{sec:BackgroundModeling}.

\clearpage
