\section{Data and Monte Carlo Samples}
\label{sec:Samples}

\subsection{Data Samples}
\label{ssec:DataSamples}

This analysis uses the first part of the ATLAS Run~2 dataset
consisting of $\sqrt{s} = 13$~TeV collision data taken with stable LHC
beams in the years 2015 and 2016.  The ATLAS muon and tracking
detectors, which are essential for the reconstruction of the \Bds\ and
\Bpm\ mesons, are required to be fully operational.  These
requirements yield an integrated luminosity of $\approx
39\;\mathrm{fb}^{-1}$.

For the \Bmumu signal channel a region of 360~MeV width around the
\Bs\ mass \yel[This sentence will need
to be adjusted once we have the reprocessed DAODs.]
{is omitted} during the analysis development to avoid any bias 
(``blinding'')\footnote{Future footnote text: Only once the unblinding
  decision will be given we will decrypt the blinded mass values using
  our unblinding key.  This procedure avoids the time needed to re-run the
  derivation on data once we are allowed to unblind while it still
  keeps us from accidentally looking at the mass values in the
  blinded region.}.

The derived AOD (DAOD) samples used in this analysis are produced
by the BPHY8 derivation format:\skp[10]\com{
  Please note that the p-tags of these data samples will change again
  as we reprocess our data DAODs to include the mass values from the
  blinded region in a blinded/ encrypted way.}
\begin{itemize}
\item Data 2015, physics\_Main stream:\\
  data15\_13TeV.period[$P$
  ].physics\_Main.PhysCont.DAOD\_BPHY8.grp15\_v01\_p3372\\
  with periods $P$  A, C, D, E, F, G, H, J.
\item Data 2106, physics\_Main stream:\\
  data16\_13TeV.period[$P$].physics\_Main.PhysCont.DAOD\_BPHY8.grp16\_v01\_p3372\\
  with periods $P$ A, B, C, D, E, F, G, I, K, L.
\item Data 2016, physics\_BphysDelayed stream:\\  
  data16\_13TeV.period[$P$].physics\_BphysDelayed.PhysCont.DAOD\_BPHY8.grp16\_v01\_p3372
  \\
  with periods $P$ D, E, F, G, I, K, L.
\end{itemize}


\subsection{Monte Carlo Samples}
\label{ssec:MCSamples}

Simulated Monte Carlo data samples are required for most of the analysis
steps. Dedicated MC data samples were produced, see
Table~\ref{tab:MCsamples}.

For each sample, the number of events generated is given together
with the details on the generation: \Pythia 8B plus EvtGen is used
for most samples except for the \Bhh\ peaking background and
the \bbmumuX\ continuum background channels where only \Pythia 8B is used.
EvtGen is used for the reference channels including a $J/\psi$ in the
final state in order to correctly account for the $J/\psi$
polarisation
effects.
The \BsJpsiPhi\ control channel, using \Pythia 8B and Photos is taken
from the samples produced for the \BsJpsiPhi\ analysis
within the {\it $J/\psi$} analysis subgroup: the sample is generated
flat from the angular point of view
and maps are used to obtain the correct angular distribution.
Most of the samples are processed with Atlfast-II that employs
the fast detector simulation for the calorimeter by means of
parameterisations of the longitudinal and lateral energy profile,
while the muon and tracking parts are fully simulated.
Due the less accurate simulation of calorimetry in Atlfast-II,
this cannot be used to estimate the muon fake rates, so
the \Bhh\ peaking background are
processed with the full simulation in order to have an accurate
description of the hadronic contributions.


\begin{table}[h]
  \begin{center}
    \hspace*{-1.5cm}
    \begin{tabular}{|l|l|r|l|l|}
        \hline
        Channel & Type & \multicolumn{1}{|l|}{Events}
        & Generator & Simulation \\
        \hline
        \Bsmumu & signal & 1,000,000 & \Pythia 8B + EvtGen & Atlfast-II
        \\
        \Bmumu & signal & 1,000,000 & \Pythia 8B + EvtGen & Atlfast-II
        \\
        \BpKpJpsi\ with \JpsiMuMu & reference & 1,997,000 &
        \Pythia 8B + EvtGen & Atlfast-II \\
        \BmKmJpsi\ with \JpsiMuMu & reference & 1,999,500 &
        \Pythia 8B + EvtGen & Atlfast-II \\
        \BpPipJpsi\ with \JpsiMuMu & reference & 498,000 &
        \Pythia 8B + EvtGen & Atlfast-II \\
        \BmPimJpsi\ with \JpsiMuMu & reference & 500,000 &
        \Pythia 8B + EvtGen & Atlfast-II \\
        \BsJpsiPhi\ with \JpsiMuMu, \Phikk & control & 5,000,000 &
        \Pythia 8B + Photospp & Atlfast-II \\
        \Bhh & peaking bkg. & 5,000,000 &
        \Pythia 8B & full simulation \\
        \BsKmMupNu & part. rec. bkg. & 250,000 &
        \Pythia 8B + EvtGen & Atlfast-II \\
        \BsPimMuPNu & part. rec. bkg. & 500,000 &
        \Pythia 8B + EvtGen & Atlfast-II \\
        \LPMuNu & part. rec. bkg. & 250,000 &
        \Pythia 8B + EvtGen & Atlfast-II \\
        \bbJpsiX with \JpsiMuMu & cont. bkg. & 10,000,000 &
        \Pythia 8B + EvtGen & Atlfast-II \\
        \bbmumuX & cont. bkg. & 200,000,000 &
        \Pythia 8B & Atlfast-II \\
        \hline
    \end{tabular}
    \caption{Monte Carlo data samples for signal, reference, control and
      background channels.  The background channels are sub-divided
      into peaking background (peaking bkg.),
      partially reconstructed background (part. rec. bkg.) and
      continuum background (cont. bkg.).
    }
    \label{tab:MCsamples}
  \end{center}
\end{table}

Further details on the datasets are given
in Table~\ref{tab:MCAODsamples} (for AODs) and Table~\ref{tab:MCDAODsamples}
(for DAODs). \com{Please note that the DAOD datasets will need
  adjustment once we will have produced the final round of MC DAOD.}

\begin{table}[h]
  \centering
  \footnotesize
  \begin{adjustbox}{angle=90}
    \begin{tabular}{|l|r|l|}
      \hline
      Channel & \multicolumn{1}{|c|}{\#events}  & Dataset \\
      \hline
      Bsmumu & 1,000,000 &
    mc16\_13TeV.300426.Pythia8BEvtGen\_A14\_CTEQ6L1\_Bs\_mu3p5mu3p5.merge.AOD.e4889\_e5984\_a875\_r9364\_r9315 \\ 
Bdmumu  &  1,000,000 &
mc16\_13TeV.300430.Pythia8BEvtGen\_A14\_CTEQ6L1\_Bd\_mu3p5mu3p5.merge.AOD.e4889\_e5984\_a875\_r9364\_r9315
\\
BpJpsiKp	  &  1,997,000 & mc16\_13TeV.300404.Pythia8BEvtGen\_A14\_CTEQ6L1\_Bp\_Jpsi\_mu3p5mu3p5\_Kp\_BMassFix.merge.AOD.e4862\_e5984\_a875\_r9364\_r9315 \\
BmJpsiKm  &  1,999,500 &
mc16\_13TeV.300405.Pythia8BEvtGen\_A14\_CTEQ6L1\_Bm\_Jpsi\_mu3p5mu3p5\_Km\_BMassFix.merge.AOD.e4862\_e5984\_a875\_r9364\_r9315
\\
BsJpsiPhi  &  5,000,000 &
mc16\_13TeV.300438.Pythia8BPhotospp\_A14\_CTEQ6L1\_Bs\_Jpsimu3p5mu3p5\_phi.merge.AOD.e4922\_e5984\_a875\_r9364\_r9315 \\
BpJpsiPip  &  498,000 &
mc16\_13TeV.300406.Pythia8BEvtGen\_A14\_CTEQ6L1\_Bp\_Jpsi\_mu3p5mu3p5\_Pip\_BMassFix.merge.AOD.e4862\_e5984\_a875\_r9364\_r9315 \\
BmJpsiPim  &  500,000 & mc16\_13TeV.300437.Pythia8BEvtGen\_A14\_CTEQ6L1\_Bm\_Jpsi\_mu3p5mu3p5\_Pim\_BMassFix.merge.AOD.e4862\_e5984\_a875\_r9364\_r9315  \\
Bhh  &  500,0000 &
mc16\_13TeV.300431.Pythia8B\_A14\_CTEQ6L1\_B\_hh.merge.AOD.e4889\_e5984\_s3126\_r9364\_r9315
\\
BsKmunu  &  250,000 &
mc16\_13TeV.300432.Pythia8BEvtGen\_A14\_CTEQ6L1\_Bs\_K3p5mu3p5nu.merge.AOD.e4720\_e5984\_a875\_r9364\_r9315
\\
BdPimunu  &  500,000 &
mc16\_13TeV.300433.Pythia8BEvtGen\_A14\_CTEQ6L1\_Bd\_pi3p5mu3p5nu.merge.AOD.e4720\_e5984\_a875\_r9364\_r9315 \\
LbPmunu  &  250,000 & mc16\_13TeV.300434.Pythia8BEvtGen\_A14\_CTEQ6L1\_Lambda0b\_p3p5mu3p5nu.merge.AOD.e4720\_e5984\_a875\_r9364\_r9315 \\
bbmumu  &  49,999,000 & mc16\_13TeV.300306.Pythia8B\_A14\_CTEQ6L1\_bb\_mu3p5mu3p5\_Py8RepDec.merge.AOD.e4911\_e5984\_a875\_r9364\_r9315 \\
bbJpsimumu &  10,000,000 & mc16\_13TeV.300203.Pythia8BPhotospp\_A14\_CTEQ6L1\_bb\_Jpsimu3p5mu3p5.merge.AOD.e4889\_e5984\_a875\_r9364\_r9315 \\
bbmumuX  & 200,000,000 &
mc16\_13TeV.300307.Pythia8B\_A14\_CTEQ6L1\_bb\_mu3p5mu3p5\_Py8RepDec\_4to6p5GeV.merge.AOD.e6179\_e5984\_a875\_r9364\_r9315 \\
\hline
    \end{tabular}
  \end{adjustbox}
  \caption{Monte Carlo data samples used (AOD).}
  \label{tab:MCAODsamples}
\end{table}

\begin{table}[h]
  \centering
  \scriptsize
  \begin{adjustbox}{angle=90}
    \begin{tabular}{|l|r|l|}
      \hline
      Channel & \multicolumn{1}{|c|}{\#events}  & Dataset \\
      \hline
      Bsmumu  &  918,488 & mc16\_13TeV.300426.Pythia8BEvtGen\_A14\_CTEQ6L1\_Bs\_mu3p5mu3p5.deriv.DAOD\_BPHY8.e4889\_e5984\_a875\_r9364\_r9315\_p3371 \\
Bdmumu  &  913,507 &
mc16\_13TeV.300430.Pythia8BEvtGen\_A14\_CTEQ6L1\_Bd\_mu3p5mu3p5.deriv.DAOD\_BPHY8.e4889\_e5984\_a875\_r9364\_r9315\_p3371
\\
BpJpsiKp  &  1,610,177 & mc16\_13TeV.300404.Pythia8BEvtGen\_A14\_CTEQ6L1\_Bp\_Jpsi\_mu3p5mu3p5\_Kp\_BMassFix.deriv.DAOD\_BPHY8.e4862\_e5984\_a875\_r9364\_r9315\_p3371 \\
BmJpsiKm  &  1,617,566 & mc16\_13TeV.300405.Pythia8BEvtGen\_A14\_CTEQ6L1\_Bm\_Jpsi\_mu3p5mu3p5\_Km\_BMassFix.deriv.DAOD\_BPHY8.e4862\_e5984\_a875\_r9364\_r9315\_p3371 \\
BsJpsiPhi  &  3,555,942 & mc16\_13TeV.300438.Pythia8BPhotospp\_A14\_CTEQ6L1\_Bs\_Jpsimu3p5mu3p5\_phi.deriv.DAOD\_BPHY8.e4922\_e5984\_a875\_r9364\_r9315\_p3371 \\
BpJpsiPip  &  399,607 & mc16\_13TeV.300406.Pythia8BEvtGen\_A14\_CTEQ6L1\_Bp\_Jpsi\_mu3p5mu3p5\_Pip\_BMassFix.deriv.DAOD\_BPHY8.e4862\_e5984\_a875\_r9364\_r9315\_p3371 \\
BmJpsiPim  &  405,710 & mc16\_13TeV.300437.Pythia8BEvtGen\_A14\_CTEQ6L1\_Bm\_Jpsi\_mu3p5mu3p5\_Pim\_BMassFix.deriv.DAOD\_BPHY8.e4862\_e5984\_a875\_r9364\_r9315\_p3371 \\
Bhh  & 4,002,651& mc16\_13TeV.300431.Pythia8B\_A14\_CTEQ6L1\_B\_hh.deriv.DAOD\_BPHY8.e4889\_e5984\_s3126\_r9364\_r9315\_p3371 \\
BsKmunu  &  15,786 & mc16\_13TeV.300432.Pythia8BEvtGen\_A14\_CTEQ6L1\_Bs\_K3p5mu3p5nu.deriv.DAOD\_BPHY8.e4720\_e5984\_a875\_r9364\_r9315\_p3371 \\
BdPimunu  &  30,932 & mc16\_13TeV.300433.Pythia8BEvtGen\_A14\_CTEQ6L1\_Bd\_pi3p5mu3p5nu.deriv.DAOD\_BPHY8.e4720\_e5984\_a875\_r9364\_r9315\_p3371  \\ 
LbPmunu  &  15,768 & mc16\_13TeV.300434.Pythia8BEvtGen\_A14\_CTEQ6L1\_Lambda0b\_p3p5mu3p5nu.deriv.DAOD\_BPHY8.e4720\_e5984\_a875\_r9364\_r9315\_p3371 \\
bbmumuX  &  14,666,736 & mc16\_13TeV.300306.Pythia8B\_A14\_CTEQ6L1\_bb\_mu3p5mu3p5\_Py8RepDec.deriv.DAOD\_BPHY8.e4911\_e5984\_a875\_r9364\_r9315\_p3371 \\
bbJpsimumuX &  6,807,750 & mc16\_13TeV.300203.Pythia8BPhotospp\_A14\_CTEQ6L1\_bb\_Jpsimu3p5mu3p5.deriv.DAOD\_BPHY8.e4889\_e5984\_a875\_r9364\_r9315\_p3371 \\
      \hline
    \end{tabular}
  \end{adjustbox}
  \caption{Derived Monte Carlo data samples used (DAOD).}
  \label{tab:MCDAODsamples}
\end{table}

\clearpage
