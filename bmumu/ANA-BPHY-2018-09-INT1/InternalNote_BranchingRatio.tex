\section{BranchingRatio}
\label{sec:BranchingRatio}

The \Bsmumu\ branching fraction is obtained by means of the
formula~\ref{eq:BRFormula}. We list below the various inputs
appearing in this formula starting from the ones we need to
consider from external sources and then detailing the ingredients
we collected through the various steps of this analysis.

\begin{itemize}
    \setlength{\itemsep}{0pt}%
    \setlength{\parskip}{0pt}%
\item The external inputs needed are the branching fraction for the
reference channel that is obtained from the PDG ~\cite{PDG2014}
as the product of \yel[possibly to be updated]{ $\BR(\BpmKpmJpsi)=(1.027 \pm 0.031) \times 10^{-3}$ 
and $\BR(\Jpsi\to\mumu)=(5.961 \pm 0.033)\%$.  }
The relative hadronisation probability $f_u/f_s$ is taken from 
the best experimental result~\cite{lhcbfsfu14} available:
\yel[possibly to be updated]{$f_s/f_d = 0.259 \pm 0.015$ using $f_d/f_u=1$. }

The product of these external inputs gives:
\[
\mathcal{F}_{\mathrm{ext}} = \BR (\BpmKpmJpsi\! \to \mumu K^\pm)\times\frac{f_{u}}{f_{s}} = (2.36 \pm 0.15) \times 10^{-4} 
\]
which corresponds to a relative uncertainty of $6.6\%$

\item Number of signal events, $N_{\mumu}$: at this blinded stage we consider the expected
number of events as ... derived from the SM branching ratio and the error from
the default fit as ..... The relative statistical uncertainty is about .....

\item Efficiency weighted number of events for the reference channel ( ${D}_{\mathrm{norm}}$ ).
%we can recall how this enters in the master formula in the denumerator:
%\begin{eqnarray}
%\mathcal{D}_{\mathrm{norm}} = \sum_k N^k_{J/\psi K^\pm} \alpha_k \left(\theInvRho \right)^{k} = 
%\sum_k \frac{N^k_{J/\psi K^\pm} \alpha_k}{\left(\theRho \right)^{k}}
%\label{for:denominator}
%\end{eqnarray}
%over the four categories $k=N1, N2, N3, 2011$. The second equality highlights the
%fact that efficiency ratio needed is the inverse of the one mentioned in
%Sec.~\ref{sec:eff}. Table~\ref{tab:brinputs} summarises the inputs needed
%for this calculation: the $B^\pm$ yields are the same as given in
%Table~\ref{tab:bplusyieldresults} with the statistical and systematic errors
%summed in quadrature and the efficiency ratios are the same as given in
%Table~\ref{tab:effsyst} (thus entering in the denominator within the sum above).
%The $\alpha_k$ factors are coming from the category-by-category ratios of the
%total luminosities in Tables~\ref{tab:BzeroLumi} and~\ref{tab:BplusLumi}.
%The total normalisation factor $\mathcal{D}_{\mathrm{norm}}$ is:
%\[
%\mathcal{D}_{\mathrm{norm}} = \sum_k \frac{N^k_{J/\psi K^\pm} \alpha_k}{\left(\theRho \right)^{k}}
%= (2.75 \pm 0.14) \times 10^{6}
%\]
%where the relative uncertainty is about $5\%$.
\end{itemize}

Putting together the three terms, the branching ratio is obtained as:
\[
  \BR (\Bmumu)= \frac{\mathcal{F}_{\mathrm{ext}} \times N_{\mumu}} {\mathcal{D}_{\mathrm{norm}}} = (.... \pm ....) \times 10^{-9}
\]
where the relative uncertainty is about ......

A two-dimensional Neyman construction~\cite{Neyman} based on likelihood
ratio ranking is used to estimate the 68.3\%, 95.5\% and 99.7\% confidence
level regions for the combined measurement of $\cal{B}$$(B^0_{s} \to \mu^+ \mu^-)$
and $\cal{B}$$(B^0 \to \mu^+ \mu^-)$. Pseudo-MC experiments are used in
the Neyman construction procedure and to verify the coverage.


\clearpage
