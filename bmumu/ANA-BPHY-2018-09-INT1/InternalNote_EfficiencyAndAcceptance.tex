\section{EfficiencyAndAcceptance}
\label{sec:EfficiencyAndAcceptance}

Acceptances and efficiencies for both reference and signal channels enter in
Eq.~\ref{eq:BRFormula} through their ratio \theRho.
\theRho\ is evaluated using \Bsmumu\ and \BpKpJpsi\ signal MC samples, 
after applying to both MC samples the GLC and data driven corrections (see Section~\ref{sec:MontecarloTuning}). 

For the sake of clarity, we decided to split each component of \theRho\ into two separate
terms defined as the product of a pure acceptance term $A$ and a pure efficiency term
$\epsilon$, as stated in Eq.~\ref{eq:BRFormula}. \theRho\ is then defined as:

\begin{equation}
\theRho = \frac{A_{\BpKpJpsi} \times \epsilon_{\BpKpJpsi}}{A_{\Bsmumu} \times \epsilon_{\Bsmumu}}
\label{eq:eff_ratio}
\end{equation}

More precisely, the definition of the $A$ and $\epsilon$ terms can be summarised as follows:
\begin{itemize}
\item $A$ takes into account the loss in acceptance, with respect
to the chosen phase-space fiducial volume, due to the MC generator
cuts applied on the final state particles (the two muons for both
channels and the kaon for the reference channel only) in order
to produce the MC samples used in the analysis. 
The fiducial volume is defined as $\pt^B>8.0$ GeV and $\left| \eta_B\right|<2.5$,
while the final state particle cuts are: $\pt^{\mu}>4.0$ GeV and
$\left| \eta_{\mu}\right|<2.5$ for muons and  $\pt^{K}>1.0$ GeV and
$\left| \eta_{K}\right|<2.5$ for the kaon. The $A$ term is defined
as the ratio between the number of events passing the final state
particle cuts and the number of events in the fiducial volume.
This term has been evaluated using a specific "un-biased" MC sample
where only the phase-space fiducial volume cuts on the $B$ meson
were applied and it is different between the reference and the
signal channels. 
Data driven corrections to the $\pt^B$, $\left| \eta_B\right|$ distribution are applied to the MC sample, both in the numeration and the denominator, in order to reproduce the spectrum observed in data.
in The kinematic cuts are applied to the various
particles only at the truth level. %The terms have been computed
%separately for both 2011 and 2012 MC productions. 
The values of the $A$ term are reported in Table~\ref{tab:Aterm}.\textbf{They refer to Run1 analysis and are here just to give an idea of the order of magnitude of values and uncertainties}.
\item $\epsilon$ takes into account all the reconstruction effects
(selection cuts, trigger efficiency, reconstruction efficiency)
affecting the two channels. It is defined as the ratio between the
number of events passing the final selection cuts listed in
Section~\ref{sec:Preselection} including the cut on the continuum BDT once fixed(listed in Section~\ref{sec:ContinuumBDT} ) %and the $B^+$ isolation reweighting
%procedure described in Section~\ref{subsec:eff} 
and the number of events
passing the final state particle kinematic cuts applied to the appropriate
truth level quantities. Both terms are evaluated using the simulated MC
samples available. %and they have been computed separately for the three
%trigger categories considered in 2012 analysis and for 2011 MC production. 
Since this term involves reconstructed quantities, QLC (see Section~\ref{subsec:QLC}) and data driven
corrections (see Section~\ref{sec:ddw}) have been applied to both the numerator (to reconstructed
quantities) and the denominator (to truth quantities) in order to
properly take into account the effects of a modified $p_T$ and
$\left| \eta \right|$ spectra of the $B$-meson. Trigger efficiency
corrections, that will be taken from data and will appear as Scale Factors to be applied to MC, will be added as well once available (see Section~\ref{sec:Trigger}). 
They will be applied to the numerator only. 
%In addition, the relative luminosity of the three trigger categories
%in 2012 MC simulated samples has been rescaled to match those measured in data.
\end{itemize}


\begin{table}[htbp]
\begin{center}
\begin{tabular}{|c|c|}
\hline
$A_{\BpKpJpsi}^{2012}$ & 0.0865 $\pm$ 0.0002 \\
\hline
$A_{\Bsmumu}^{2012}$ & 0.2902$\pm$ 0.0005 \\
\hline
$A_{\BpKpJpsi}^{2011}$ & 0.0819 $\pm$ 0.0003 \\
\hline
$A_{\Bsmumu}^{2011}$ & 0.3035 $\pm$ 0.0005 \\
\hline
\end{tabular}
\caption{\label{tab:Aterm}$A$-terms for $B^+$ and $B^0_s$ channels for 2012
and 2011 MC samples. The statistical uncertainty due to the finite size
of the simulated samples is also reported.} 
\end{center}
\end{table}

The values for the $A \times \epsilon$ and \theRho\ terms are summarised in
Table~\ref{tab:effsyst}, together with the statistical and systematic
uncertainties described in the next section. The statistical uncertainties
come from the finite statistic available for the simulated samples used
in the analysis. \textbf{They refer to Run1 analysis and are here just to give an idea of the order of magnitude of values and uncertainties}.
The sources of systematic uncertainties affecting the values reported in
Table~\ref{tab:effsyst} are described in Section~\ref{subsec:eff}.
%In the evaluation of $A \times \epsilon$ for both signal and
%reference-channel the events passing the baseline and additional
%selection criteria are normalised to the events generated in a
%fiducial phase-space volume with $\pt^B>8.0$ GeV and $\left| \eta_B\right|<2.5$.
%As the master formula~\ref{eq:BRFormula} has been modified to
%take into account possible differences in the three trigger
%categories, the $A \times \epsilon$ values for signal and
%reference channels and their ratios are now calculated separately
%in each trigger category and are summarised in Table~\ref{tab:effsyst}.

%\input{BMuMuCOMM_effandacc_table}
\begin{table}[htbp]
\begin{center}
\begin{tabular}{|c|c|c|c|}
\hline
channel & $\epsilon$ & $A \times \epsilon$ & \theRho \\
\hline
\hline
%\multirow{2}{24mm}{N1} & $B^+$ & 0.0040 & 0.00035$\pm$1.26\%$\pm$ 12.9\% & \multirow{2}{55mm}{0.189 $\pm$2.5\%~(stat)$\pm$11.5\%~(syst)}\\ 
%\cline{2-3} & \Bs\ & 0.0063 & 0.00183$\pm$ 2.18\% $\pm$ 12.1\% & \\
%\hline
%\hline
%\multirow{2}{24mm}{N2} & $B^+$ & 0.0104 & 0.00090$\pm$0.82\%$\pm$7.3\% & \multirow{2}{55mm}{0.226$\pm$1.78\%~(stat)$\pm$6.4\%~(syst)}\\ 
%\cline{2-3}  & \Bs\  & 0.0137 & 0.00398$\pm$1.78\% $\pm$5.4\% & \\
%\hline
\hline
$B^+$ & 0.0928 & 0.0080$\pm$0.28\% $\pm$14.2\% & \multirow{2}{55mm}{0.180$\pm$0.56\%~(stat)$\pm$5.2\%~(syst)}\\ 
$B^0_s$ & 0.1522 & 0.0441$\pm$0.49\% $\pm$10.3\% & \\
\hline
%\hline
%\multirow{2}{24mm}{Inclusive} & $B^+$ & 0.1072 & 0.0090$\pm$0.26\%$\pm$12.8\% & \multirow{2}{55mm}{0.1790$\pm$0.53\%~(stat)$\pm$6.7\%~(syst)}\\ 
%\cline{2-3}  & \Bs\  & 0.1721 & 0.0507$\pm$0.47\% $\pm$ 9.0\% & \\
%\hline
%\hline
%\multirow{2}{24mm}{2011} & $B^+$ & 0.1108 & 0.0091$\pm$0.25\%$\pm$6.4\% & \multirow{2}{55mm}{0.156$\pm$1.02\%~(stat)$\pm$5.5\%~(syst)}\\ 
%\cline{2-3}  & \Bs\  & 0.1968 & 0.0598$\pm$0.97\% $\pm$ 6.6\% & \\
%\hline
\end{tabular}
\caption{\label{tab:effsyst}$\epsilon$, $A \times \epsilon $ and \theRho\ values for $B^+$
and  $B^0_s$ channels for 2012 (split into the three trigger categories) and 2011 samples.
Where present, the first uncertainty is statistical and second one is systematic.} 
\end{center}
\end{table} 

\subsection{Systematic uncertainties on \theRho\ }
\label{subsec:eff}
The systematic uncertainty affecting \theRho\ comes mainly from two sources:
the uncertainty related to the corrections (GLC, data driven and trigger
efficiencies) and the residual discrepancy between data and MC samples
after the application of these corrections. The total systematic uncertainty
quoted in Table~\ref{tab:effsyst} is the sum in quadrature of these two components.

%The input numbers used for calculation of the $A \times \epsilon$
%values and their ratios are detailed in appendix~\ref{app:effsyst}
%in Table~\ref{app:tab:effsyst}. 

The first source of systematic uncertainty considered originates
from the uncertainty of the GLC, DDW and trigger efficiency corrections.
For evaluation of the effect a toy study has been performed by varying
the corrections within their statistical uncertainties and recomputing
each term of the \theRho\ ratio after each toy. The RMS of the values
obtained in each toy has been quoted as the systematic uncertainty for
the various quantities. %The uncertainties on \theRho\ range from 1\% to 5\% depending on the trigger category considered. For the inclusive 2012 category it has been found to be 1.5\%, while for 2011 samples it has been found to be 2.2\%.
%The results of this study are summarised
%in~\ref{app:systweight}.

Another source of systematic uncertainty arises from the discrepancies
between data and MC. The systematic uncertainty on \theRho\ was assessed
by observing the variation in the efficiency of the final selection when
re-weighting both the MC samples to the observed data for each of the 15
variables used in the continuum BDT.
Data are extracted from $B^{+}$ events after the subtraction of the
background as shown in section~\ref{sec:DataMCComparison}.  \theRho\ is recomputed after the reweighting of each variable
one at a time. The discrepancy between the values obtained with this
procedure and the central values has been considered as the systematic
uncertainty due to the specific variable mis-modelling.\\


%Only the isolation variable
%has been treated differently between reference and signal channels, due to
%the different electric charge of the two mesons. In this particular case,
%isolation weights have been extracted from \BsJpsiPhi\ data (as explained
%in Section~\ref{sec:jpsiphi}) and then applied to \Bsmumu\ simulated
%sample. \theRho\ has been recomputed after the reweighting of each variable
%one at a time. The discrepancy between the values obtained with this
%procedure and the central values has been considered as the systematic
%uncertainty due to the specific variable mis-modelling.\\
%The main uncertainties for the reference channel in 2012 came from 
%the isolation $I_{0,7}$ (-8.9\%, determined with an uncertainty of
%$\pm$~3.2\% from statistical fluctuation in the samples and in the
%background subtraction)%, the number of tracks close to the reconstructed $B$-vertex
%%$N^{close}_{trks}$ (-5.2\%) and the distance of closest approach of the closest
%%track to the $B$ candidate $DOCA_{xtrk}$ (-4.5\%). 
%For the signal, the main components were:
%$\chi^2_{\mu,xPV}$ (-4.8\%) and the isolation $I_{0,7}$ extracted
%from the \BsJpsiPhi channel (-3.8\%). 
%Given the significant impact of the isolation reweighting on the efficiency of
%the reference channel, sign of an important mis-modelling of MC sample with
%respect to data, we decided to reweight the $B^+$ MC 2012 sample to match the
%isolation distribution of the $B^+$ meson to that in data. Coherently, we have reweighted
%also the isolation in the signal MC using the weights extracted from the \BsJpsiPhi channel.
%The full list of the systematic uncertainties, after the isolation
%reweighting, for all triggers categories split into the 15 variables
%individually can be found in the Appendix~\ref{app:effsyst}.\\
%The vast majority of the systematic uncertainties obtained from this
%procedure have an impact lower than 2\% on \theRho\. However, we preferred to quote
%as a total systematic uncertainty on \theRho\ the sum in quadrature of the single
%discrepancies before the isolation reweighting.
%In Table~\ref{tab:effsystapp} we reported the impact of each systematic uncertainty
%before and after the isolation reweighting.

%For 2011 MC production, the overall agreement between data and MC was better.
%The main uncertainty on the reference channel came from the distance of
%closest approach of the closest track to the $B$ candidate
%$DOCA_{xtrk}$ (+4.8\%). For the signal, the main components were:
%the isolation $I_{0.7}$ extracted from the \BsJpsiPhi channel (-4.3\%)
%and the distance of closest approach $DOCA_{xtrk}$ (+3.8\%).
%All the systematic uncertainties obtained with this procedure have an
%impact lower than 2\% on \theRho except for the isolation $I_{0.7}$ that has an impact of 3.2\%.

%As a cross-check, the same reweighting procedure has been applied directly
%to the continuum-BDT distribution of the reference channel. The impact on
%$A \times \epsilon$ has been found to be -13.2\%. This value is in agreement
%with the 12.8\% obtained with the reweighting procedure described above on
%2012 $B^+$ MC (see Table~\ref{tab:effsyst}).
%The effect of the continuum-BDT reweighting on the signal channel efficiency
%has also been evaluated. If the continuum-BDT weights extracted from $B^+$
%data are applied, the variation on \Bs\ efficiency is found to be -14.9\%
%(leading to a variation on \theRho\ of 2.1\%), while if those extracted from
%\BsJpsiPhi\ data are applied, the effect is -5.9\% (-7.8\% on \theRho).
%Since the isolation is different between the two channels, the use of the
%same continuum-BDT weights (those extracted from $B^+$ data) would
%underestimate the total systematic uncertainty on \theRho\.
%Therefore the procedure described above (i.e. the sum in quadrature of the
%single effects on the efficiencies of the 15 variables entering in the
%continuum BDT) is found more appropriate to quote a reliable systematic
%uncertainty on \theRho\ due to data/MC discrepancies.

%The same c-BDT reweighting procedure applied to 2011 samples showed a
%discrepancy of 7.0\% in \theRho\. This number has to be compared with the sum
%in quadrature of the single effects on \theRho\ of the 15
%variables entering in the c-BDT, that gives a discrepancy of 3.4\%. In order
%to be conservative, we preferred to quote 7.0\% as systematic uncertainty on
%\theRho\ rather than the sum in quadrature of the single effects as for 2012 case.


The total systematic uncertainty on $A \times \epsilon$ and \theRho\ due
to the reweighting procedure is the sum in quadrature of the single
systematic uncertainties and it is quoted in Table~\ref{tab:effsyst}. 

%In order to perform this systematic evaluation, the variables are classified according to their relative correlations. Four main groups of variables are identified:
%\begin{itemize}
%    \setlength{\itemsep}{0pt}%
%    \setlength{\parskip}{0pt}%
%\item $1^\circ$ block: Isolation variables ($5$ variables)
%\item $2^\circ$ block: Vertex and IP related variables ($12$ variables)
%\item $3^\circ$ block: χ2(B) and DCA (2 variables)
%\item $4^\circ$ block: Kinematic variables (3 variables)
%\end{itemize}

%Each variable in one block is highly ($50\%$ or more) correlated with the others of the same block but almost uncorrelated with those belonging to another block. Considering the ranking of the variables from the continuum-BDT training, in each block the variable with higher ranking is selected:
%\begin{itemize}
%    \setlength{\itemsep}{0pt}%
%    \setlength{\parskip}{0pt}%
%\item $1^\circ$ variable: $\alpha_{2D}$ ($1^\circ$ place in the ranking)
%\item $2^\circ$ variable: $\pt^B$ ($4^\circ$ place)
%\item $3^\circ$ variable: $B$ isolation ($6^\circ$ place)
%\item $4^\circ$ variable: $\chi^2(B)$ ($9^\circ$ place)
%\end{itemize}

%In order to perform this systematic evaluation, we consider the data-MC discrepancies in each of the $4$ variables selected above and we extract independently for each of them a set of weights correcting for the discrepancies (with a similar technique used for the DDW).
%The difference in the \theRho\ found by applying these sets of weights are taken as systematics due to the residual discrepancies. The four contributions are then added in quadrature.


%For evaluation of the systematic uncertainty a toy study, a variation
%of the data-MC weights within their statistical uncertainties and
%a re-evaluation of the ratio of $A \times \epsilon$, is performed.
%This study is summarised in appendix~\ref{app:systratios}.
%Because of the significance discrepancies observed between sideband-subtracted
%data and signal MC of the reference channel, especially for vertexing variables,
%we study the effect of each variable separately. First, since the vertexing
%variables are strongly correlated, we use data-MC weights of the most prominent
%one, $L_{xy}$, for re-weighting the MC sample, factorising its effect on the ratio
%of $A \times \epsilon$, estimated to be 2\%.
%After the application of the $L_{xy}$ corrections the impact of all other variables
%but isolation on the ratio becomes small and sums up to 2.3\% (quadratic addition).
%The effect of the isolation variable, which is not affected by the $L_{xy}$
%corrections, has been studied separately. It has been observed that its behaviour
%differs for the \Bsmumu\ and \BpKpJpsi\ channels even after 
%matching the kinematic acceptance of B-mesons in both channels. 
%Presumably this is due to difference in multiplicities of charged
%tracks in the corresponding underlying event.  By comparing the
%sideband-subtracted data and the MC of the \BsJpsiPhi\ sample it has
%been observed that the isolation distributions of \Bs\ candidates agree
%within the statistics of the available sample.
%%(Figure~\ref{fig:jpsiphi} right). 
%Therefore we conservatively propagate the 6.1\% difference in BDT
%selection efficiency of $B^+$ candidates as the contribution of the
%isolation variable to systematic uncertainty of the $A \times
%\epsilon$ ratio.
%Combining the contributions in quadrature, the resulting uncertainty on
%%% $\rho=\frac{\epsilon_{J/\psi K^+} A_{J/\psi K^+}}{\epsilon_{\mu\mu}
% %% A_{\mu\mu}}$
%$\rho = \theRho$ 
%is $\Delta\rho/\rho = \pm 6.9\%$(syst), while the statistical
%uncertainty on $\rho$ due to the finite MC sample is 
%$\Delta\rho/\rho = \pm 1.8\%$. 
%
%Application of data-MC weights to the MC data changes the shape of BDT
%output distribution 
%as shown in Figure~\ref{fig:qResponseVariation} for the
%\Bsmumu\ and \BpKpJpsi\ events using data-MC weights of only
%$L_{xy}$ variable, taken as an example.
%The same plot shows the classifier
%response for \Bsmumu\ background, 
%drawn from the data sidebands, for the reference. Despite of the
%significant changes in the efficiency of the optimised cuts, 
%which one can infer from the plot, 
%these changes are highly correlated between the signal and the
%reference channel, keeping the impact on the ratio of $A \times
%\epsilon$ relatively small.  
%
%We verify on \BpmKpmJpsi\ candidates that the observed efficiency of
%the optimised selection criteria relative to the signal preselection
%is $(25.9 \pm 0.8)\%$~(stat) in the data, which 
%is compatible with ($27.0 \pm 0.1$~(stat)~$\pm 3.6$~(syst))\% estimated
%from MC.
%%3.6 = sqrt (11.3^2+6.1^2+4.1^2)

%%%%%%%%%%%%%%%%%%%%%%%%%%%%%


%\begin{figure}[h!] 
%\center
%\includegraphics[scale=0.75]{eps/bdt/bdt-bkg_sig_sigwei.eps}
%\caption{\label{fig:qResponseVariation} Distribution of the response of the
%MVA classifier for odd-numbered sidebands (leftmost black histogram),
%$B\to\mu\mu$ signal (red and blue) and \BpKpJpsi\ (magenta and green).
%The blue and green histograms are computed using the standard MC
%modelization, while the red and magenta ones are re-weighted
%using data-MC weights of $L_{xy}$ variable, taken as an example.} 
%\end{figure}



%%
%% WW, 2013-04-09: old numbers for B flow -- commented here
%% identical table moved to appendix/flow comparison.tex
%% as it was used in that particular study.
%%
%\begin{table}[htbp]
%\begin{center}
%\begin{tabular}{|c|c|c|c|}
%\hline
%category & channel & $A \times \epsilon$ [\%] & \theRho\ \\
%\hline
%\hline
%\multirow{2}{24mm}{1 bin} & $B^+$ & 1.092$\pm$0.005 & \multirow{2}{55mm}{0.262$\pm$1.9~\%(stat)$\pm$1.8~\%(syst)}\\
%\cline{2-3}
% & \Bs\ & 4.168$\pm$0.076 & \\
%\hline
%\hline
%\multirow{2}{24mm}{BB} & $B^+$ & 1.506$\pm$0.010 & \multirow{2}{55mm}{0.287$\pm$2.7~\%(stat)$\pm$2.6~\%(syst)}\\
%\cline{2-3}
% & \Bs\  & 5.253$\pm$0.138 & \\
%\hline
%\hline
%\multirow{2}{24mm}{BT} & $B^+$ & 1.057$\pm$0.011 & \multirow{2}{55mm}{0.215$\pm$3.6~\%(stat)$\pm$5.8~\%(syst)}\\
%\cline{2-3}
% & \Bs\ & 4.912$\pm$0.170 & \\
%\hline
%\hline
%\multirow{2}{24mm}{AE} & $B^+$ & 0.597$\pm$0.007 & \multirow{2}{55mm}{0.195$\pm$3.5~\%(stat)$\pm$1.5~\%(syst)}\\
%\cline{2-3}
% & \Bs\ & 3.052$\pm$0.102 & \\
%\hline
%\end{tabular}
%\caption{\label{tab:effsystB}$A \times \epsilon$ for $B^+$ and \Bs\ channels and their ratios per resolution zone for the $B$ flow.}
%\end{center}
%\end{table}


%\input{effacc_syst_table}

% DETAILED TABLE FOR SYSTEMATICS


%\clearpage
%\newpage
%\section{Studies of systematic uncertainties on the relative
%Efficiency$\times$Acceptance ratio}
%\subsection{Complete table for the Efficiency$\times$Acceptance
%ratio evaluation}
%\label{app:effsyst}
%\begin{table}[htbp]
%\scriptsize
%%\footnotesize
%\begin{center}
%\begin{tabular}{|c|c|c|c|c|c|c|}
%\hline
%Variable & Efficiencies & N1 & N2 & N3 & Inclusive 2012 & 2011\\
%\hline
%$|\alpha_{2D}|$  & $A \times \epsilon_{\Bsmumu}$ & 0.4 (0.4)& 1.7 (1.8)& -1.2 (-1.2)& -0.9 (-0.9)& 1.5\\
 %& $A \times \epsilon_{\BpKpJpsi}$ & 0.5 (0.6)& 2.5 (2.5)& -1.4 (-1.4)& -0.9 (-0.9) & 2.1\\
 %& \theRho & 0.1 (0.2)& 0.8 (0.8)& -0.2 (-0.2)& 0.1 (0.1) & 0.6\\
%\hline
%$\chi^2_{\mu,xPV}$   & $A \times \epsilon_{\Bsmumu}$ & 4.4 (4.6)& -0.8 (-0.8)& -5.5 (-5.6)& -4.8 (-4.9)& 0.3\\
 %& $A \times \epsilon_{\BpKpJpsi}$ & 3.5 (3.5)& -1.3 (-1.3)& -3.4 (-3.8)& -2.9 (-3.3)& 0.9\\
 %& \theRho & -0.9 (-1.0)& -0.5 (-0.5)& 2.2 (1.9) & 1.9 (1.7)& 0.6\\
%\hline
%$I_{0.7}$   & $A \times \epsilon_{\Bsmumu}$ & -2.9 & -4.5 & -3.8 & -3.8 & -4.3\\
 %& $A \times \epsilon_{\BpKpJpsi}$ & -4.0 & -5.7 & -9.4 & -8.9 & -1.3\\
 %& \theRho & -1.1 & -1.2 & -5.8 & -5.3 & 3.2\\
%\hline
%IP$_B^{3D}$   & $A \times \epsilon_{\Bsmumu}$ & 0.5 (0.6)& 0.3 (0.3)& 0.1 (0.1) & 0.1 (0.1) & -0.1\\
 %& $A \times \epsilon_{\BpKpJpsi}$ & -0.1 (-0.2)& 0.4 (0.4)& -0.3 (-0.4)& -0.3 (-0.3)& +0.4\\
 %& \theRho & -0.6 (-0.8)& 0.1 (0.1)& -0.4 (-0.5)& -0.4 (-0.4) & 0.5\\
%\hline
%DOCA$_{\mu\mu}$   & $A \times \epsilon_{\Bsmumu}$ & -1.1 (-1.5)& -0.4 (-0.3)& -0.5 (-0.5) & -0.5 (-0.5)& -0.3\\
 %& $A \times \epsilon_{\BpKpJpsi}$ & -1.7 (-1.8)& -1.0 (-1.1)& -0.5 (-0.6)& -0.6 (-0.7)& 0.1\\
 %& \theRho & -0.6 (-0.4)& -0.7 (-0.8)& 0.1 (-0.1)& -0.1 (-0.2)& 0.4\\
%\hline
%$L_{xy}$   & $A \times \epsilon_{\Bsmumu}$ &2.4 (2.5)& -0.2 (-0.1)& -1.3 (-1.4)& -1.0 (-1.1)& 1.1\\
 %& $A \times \epsilon_{\BpKpJpsi}$ & 6.2 (6.5)& 1.9 (2.0)& -3.2 (-3.4)& -2.4 (-2.5)& 1.2\\
 %& \theRho & 3.6 (3.9)& 2.1 (2.2)& -2.0 (-2.1)& -1.4 (-1.4)& 0.1\\
%\hline
%DOCA$_{xtrk}$   & $A \times \epsilon_{\Bsmumu}$ & -0.4 (-0.1) & 0.1 (0.1)& -3.5 (-3.4)& -3.1 (-3.0)& 3.8\\
 %& $A \times \epsilon_{\BpKpJpsi}$ & -0.8 (-0.7)& -1.3 (-1.2)& -5.0 (-4.9)& -4.5 (-4.4)& 4.8\\
 %& \theRho & -0.4 (-0.6)& -1.3 (-1.4)& -1.6 (-1.5)& -1.5 (-1.4)& 1.0\\
%\hline
%$N^{close}_{trks}$   & $A \times \epsilon_{\Bsmumu}$ & -1.9 (-1.9) & -1.1 (-1.2)& -3.6 (-3.6)& -3.3 (-3.3)& 2.0\\
 %& $A \times \epsilon_{\BpKpJpsi}$ & -2.5 (-2.5)& -2.2 (-2.2) & -5.6 (-5.8)& -5.2 (-5.3)& 2.1\\
 %& \theRho & -0.6 (-0.6)& -1.1 (-1.1) & -2.1 (-2.3)& -1.9 (-2.1)& 0.1\\
%\hline
%$|d_{0}|^{min}$ sign.   & $A \times \epsilon_{\Bsmumu}$ & 7.3 (7.0)& 1.3 (1.3)& -4.4 (-4.5) & -3.5 (-3.6)& 1.0\\
 %& $A \times \epsilon_{\BpKpJpsi}$ & 2.7 (2.9)& -0.3 (-0.2)& -2.1 (-2.4)& -1.8 (-2.0)& 1.0\\
 %& \theRho & -4.3 (-3.9)& -1.6 (-1.5)& 2.4 (2.2)& 1.8 (1.7)& -0.1\\
%\hline
%$|d_{0}|^{max}$ sign.   & $A \times \epsilon_{\Bsmumu}$ & 4.9 (5.0)& 0.1 (0.2)& -1.6 (-1.7)& -1.3 (-1.3) & 0.8\\
 %& $A \times \epsilon_{\BpKpJpsi}$ & 5.7 (5.9)& 0.8 (0.9)& -2.0 (-2.3)& -1.5 (-1.7)& 1.3\\
 %& \theRho & 0.8 (0.9)& 0.8 (0.8) & -0.4 (-0.6) & -0.2 (-0.4)& 0.5\\
%\hline
%$P_{L}^{min}$   & $A \times \epsilon_{\Bsmumu}$ & -1.4 (-1.6)& 1.1 (1.0)& -0.4 (-0.4)& -0.3 (-0.3)& 0.4\\
 %& $A \times \epsilon_{\BpKpJpsi}$ & 2.8 (2.8)& 0.3 (0.4)& 0.4 (0.3)& 0.5 (0.4)& 0.9\\
 %& \theRho & 4.3 (4.5) & -0.8 (-0.6)& 0.8 (0.8)& 0.8 (0.8)& 0.5\\
%\hline
%$\Delta R$   & $A \times \epsilon_{\Bsmumu}$ & 3.6 (3.5)& -0.5 (-0.5)& -1.9 (-2.0)& -1.6 (-1.7)& 0.7\\
 %& $A \times \epsilon_{\BpKpJpsi}$ & 3.5 (3.6)& -0.6 (-0.6)& -2.5 (-2.6)& -2.1 (-2.2)& 0.7\\
 %& \theRho & 0.1 (0.1)& -0.1 (-0.1)& -0.6 (-0.6)& -0.5 (-0.5)& 0.1\\
%\hline
%$\chi^{2}_{xy}$   & $A \times \epsilon_{\Bsmumu}$ & 3.9 (4.0)& 0.3 (0.3)& -2.8 (-3.0)& -2.3 (-2.4)& 0.9\\
 %& $A \times \epsilon_{\BpKpJpsi}$ & 5.5 (5.9)& 0.9 (0.9)& -4.1 (-4.5)& -3.3 (-3.6)& 1.1\\
 %& \theRho & 1.6 (1.8)& 0.5 (0.6)& -1.3 (-1.6)& -1.0 (-1.1)& 0.1\\
%\hline

%\hline
%\end{tabular}
%\caption{\label{tab:effsystapp}Systematic uncertainties (in \%) on $A \times \epsilon $ and \theRho\ values for $B^+$ and  \Bs\ channels for 2012 (split into the N1, N2 and N3 trigger categories and inclusive) and
%2011. In parethesis the same systematic uncertainties after the isolation reweighting described in Section~\ref{subsec:eff} is applied.} 
%\end{center}
%\end{table}

% OLD STUFF


%\subsection{Consistency check with unbiased MC}
%\label{sec:unbiasedMC}

%As discussed in detail in section~\ref{sec:glc}, in order to reduce the need
%of computing resources, the main MC samples used in this analysis have been
%generated with: (a) a selection placed on the $B$ meson decay products kinematics
%(looser than the requirements placed at reconstruction and final selection),
%and (b) a reduced phase-space for the scattering of the fundamental constituents,
%causing an appreciable distortion in the low range of the \pt\ spectrum of the
%simulated $B$ decays. 

%These procedures were correct by:
%(a) computing the $A$ terms of the acceptance, equation~\ref{eq:eff_ratio} in
%this section, through a dedicated {\it unbiased} MC sample used at generation
%({\it truth}) level; and (b) applying the GLC correction
%%quark-level correction $W_\mathrm {QL}$ 
%on the (\pt ,\eta )  distribution of the $B$ meson,
%%equation~\ref{for:qlc} in section~\ref{sec:glc}, 
%with the weight obtained also through the unbiased MC samples.

%As a cross checks of these procedures, a fraction of the events from the unbiased
%generation have been processed through full simulation. The final samples of events
%produced in this way (i.e., accepted, reconstructed and satisfying the selection
%requirements)  are much smaller than the samples generally used  throughout this
%analysis  (namely, by factors of 13 and 15 respectively in the \Bs\ and \Bp\ modes),
%but despite the statistical limitation, they could be used for a direct check
%of the $A$ terms and of the GLC %$W_\mathrm {QL}$ 
%weights with good accuracy.
%In turn, good consistency have been found in the distributions of the $B$ meson
%kinematic variables and of the discriminating variables used in the continuum-BDT selection
%for $\mu^+\mu^-$ background. Finally, the efficiency terms  $\epsilon$ for the the
%final selection have been confirmed with an accuracy of about 10\%.
%The \pt\ and \eta\ distributions from the biased and unbiased MC samples
%are compared in figure~\ref{fig:BiasedUnBiased}. In this figure,
%the %$W_\mathrm {QL}$ 
%GLC weights %(table~\ref{tab:qlc} in section~\ref{sec:glc})
%are not applied to the biased sample, causing the deviations in the low range
%of the \pt\ spectrum. 

%\begin{figure} [htb]
  %\begin{center}
 %\hspace{-1cm}
   %\includegraphics[width=0.5\textwidth]{eps/comp/pT_MC12_Bs_BiasedUnBiased.eps}
  %\includegraphics[width=0.5\textwidth]{eps/comp/BDT_MC12_Bs_BiasedUnBiased.eps}\\
  %\hspace{-1cm}
 %\includegraphics[width=0.5\textwidth]{eps/comp/pT_MC12_Bp_BiasedUnBiased.eps}
 %\includegraphics[width=0.5\textwidth]{eps/comp/BDT_MC12_Bp_BiasedUnBiased.eps}
 %\caption { \label{fig:BiasedUnBiased} Comparisons between the biased (black points)
   %and unbiased (red points) MC samples, after reconstruction and selection,
   %before applying the continuum-BDT selection.
   %{\it Top-left}: \pt\ distributions for \Bsmumu; the region below 13~GeV shows the
   %bias due to the reduced phase-space at quark level, present and not corrected in
   %the biased sample. 
   %{\it Top-right}: comparison of the continuum-BDT distributions for \Bsmumu. 
   %{\it Bottom-left} and {\it -right}: the as same above for \BpKpJpsi .
 %} 
 %\end{center}
 %\end{figure}

%\subsection{Study of Efficiency versus continuum-BDT Bins}
%\label{sec:BDTbineff}

%As explained in Sec.~\ref{sec:sigfits}, the extraction of the \Bs\ signal is done
%with a fit performed in three contiguous intervals of the continuum BDT.
%The low edge of bin-1 is equal to the BDT threshold used for the final selection
%of candidates in the \Bsmumu\ channel and in the reference \BpKpJpsi\ and \BsJpsiPhi\ 
%channels.
%The separation values defining bin-2 and bin-3 are chosen so that for \Bsmumu\
%the nominal efficiencies of the three bins are equal. 
%The uncertainty in the actual efficiency of each bin can be estimated with the
%same method used for the efficiency of the final selection: for each discriminating
%variable, and for events passing the final selection, the distribution observed is
%\BpKpJpsi\ data is compared to the corresponding one from the MC, and the ratio
%is used to reweight the \Bsmumu\ MC distribution and compute the corresponding change
%in efficiency. 
%The deviations associated to the different discriminating variables are combined.
%Alternatively, the reweighting can be performed using the distributions of the
%continuum-BDT output, rather than those of each discriminating variables.
%Finally, the procedure is repeated using as reference the \BsJpsiPhi\ channel.


%% using the study can be performed using the BDT distribution.  
%%and the ratios used to weight the distribution for 
%%the distributions of the discriminating values in data an MC is compared for \BpKpJpsi candidates.  
%%The analysis technique to extract the signal yield is based on the knowledge
%%of the efficiency values in three different continuum-BDT bins
%%Using a similar technique to that used in the systematic studies on the
%%efficiency determination for the we can also study the uncertainty on the efficiency of the \Bs\ 
%%signal in each f the three bins 
%%as a function of continuum BDT values. As explained in Sec.~\ref{sec:sigfits},
%%the analysis technique to extract the signal yield is based on the knowledge
%%of the efficiency values in three different continuum-BDT bins.
%%Thus we need to estimate the uncertainty associated to this assumption.

%Tables~\ref{tab:systBDTbin-Bp} and \ref{tab:systBDTbin-BsJpsiphi} show the result of this study. 
%The $J/\psi K+$ and $J/\psi \phi$ channels provide comparable results in the absolute values
%of the efficiency variations, and in their quadratic combination, while the $J/\psi K+$ provides a 
%larger effect than $J/\psi \phi$ if the deviations are summed linearly. 

%Figure~\ref{fig:systBDTbins} shows the continuum-BDT distributions observed in the two channels, superimposed to 
%those for the corresponding MC. In both cases, the ratio of data to MC is described by a linear fit,
%with appreciable negative slopes of similar magnitude. 

%Table~\ref{tab:systBDTbin-BDTclassifier} shows the effect of reweighting the \Bsmumu MC according
%to the continuum-BDT distribution, following the pattern observed for \BpKpJpsi and \BsJpsiPhi.  
%The two reference channels provide a similar pattern, with about -10\% variation in bin-3,  
%a negligible change in bin-2, and a compensating variation of +10\% in bin-1. 
%Through the use of toy experiments, we estimate that such variation of bin efficiencies  would produce 
%a deviation of about -7\% in the fit result. 

%This last method, based of the continuum-BDT distributions in the reference channels, is chosen to assign an
%uncertainty to the values of the efficiencies in the three bins.
%In the likelihood fit described in Sec.~\ref{sec:sigfits}, the uncertainties are described with 
%Gaussian constraints corresponding to $\pm$10\% (relative) in bin-3, $\pm$1\% in bin-2,
%and with the efficiency of bin-1 constrained to balance the others.
%Coherent uncertainties in the total efficiency are described separately, and follow the 
%estimate described above in section~\ref{subsec:eff}.

%%Using the weights from B+\\
%%classify the variables reweighting as a function of their impact in the last c-BDT bin\\
%%\\
%%we can take this into account in the fit with Gaussian constraints on the fraction of signal
%%events in each bin (nominal value at 33.3\%\\

%\renewcommand\arraystretch{1.1}
%\begin{table}[ht!]
  %\centering
  %\begin{tabular}{l|c|c|c}
    %\hline
    %& \multicolumn{3}{c}{ absolute (relative) variation w.r.t. 33.3\% [\%] }\\
    %variable & bin 1 & bin 2 & bin 3 \\
    %\hline
    %B isolation $I_{0.7}$ & +1.5 (+4.5) & +1.2 (+3.6) & -2.6 (-7.8) \\
    %$|d_{0}|^{max}$ sig. & +1.7 (+5.1) & +0.2 (+0.6) & -1.9 (-5.7) \\
    %$|d_{0}|^{min}$ sig. & +1.9 (+5.7) & +0.2 (+0.6) & -1.9 (-5.7) \\
    %$\chi^2_{\mu,xPV}$ & +1.8 (+5.4) & 0.0 (0.0) & -1.8 (-5.4) \\
    %$\log[\chi^{2}_{\mathrm{PV,SV}}]_{xy}$ & +1.0 (+3.0) & +0.4 (+1.2) & -1.3 (-3.9) \\
    %DOCA$_{xtrk}$ & -0.5 (-1.5) & +0.1 (+0.3) & +0.5 (+1.5) \\
    %$N^{close}_{trks}$ &  -0.3 (-0.9) & -0.1 (-0.3) & +0.5 (+1.5) \\
    %$\Delta\phi(\mu\mu)$ & +0.6 (+1.8) & 0.0 (0.0) & -0.5 (-1.5) \\
    %$L_{xy}$ &  -1.0 (-3.0) & +0.7 (+2.1) & +0.4 (+1.2) \\
    %$\Delta{R}$ & -0.4 (-1.2) & +0.2 (+0.6) & +0.3 (+0.9) \\
    %$P_{L}^{min}$ & +0.3 (+0.9) & -0.1 (-0.3) & -0.1 (-0.3) \\
    %$|\alpha_{2D}|$ & -0.1 (-0.3) & +0.1 (+0.3) & +0.1 (+0.3) \\
    %IP$_B^{3D}$ & +0.1 (+0.3) & -0.1 (-0.3) & +0.1 (+0.3) \\
    %DOCA$_{\mu\mu}$ & +0.2  (+0.6) & -0.1 (-0.3) & 0.0 (0.0) \\
    %\hline
    %quadratic sum   & 3.9 (11.6)  & 1.5 (4.5)  & 4.5 (13.5) \\
    %\hline
    %linear sum   & +6.8 (+20.4)  & +2.8 (+8.4)  & -8.2 (-24.6) \\
    %\hline
  %\end{tabular}
  %\caption{Variation of the continuum-BDT efficiency in the three bins used
    %for extracting the \Bs\ signal. For each discriminating variable,
    %the ratio of the distribution for \BpKpJpsi\ in data and simulation
    %is used to reweight the MC distribution for \Bsmumu. 
    %The table shows absolute and relative variation in the nominal efficiency
    %of each BDT bin.}
%\label{tab:systBDTbin-Bp}
  %\end{table}


%%weight = -0.930733*BDT+1.34205\\
%%signal efficiency in each BDT bin:\\   this is J.psi K+ , latest BDT
%%bin 1 = 37.0% \\
%%bin 2 = 33.2% \\
%%bin 3 = 29.7%\\

%\renewcommand\arraystretch{1.1}
%\begin{table}[ht!]
  %\centering
  %\begin{tabular}{l|c|c|c}
    %\hline
    %& \multicolumn{3}{c}{ absolute (relative) variation w.r.t. 33.3\% [\%] }\\
    %variable & bin 1 & bin 2 & bin 3 \\
    %\hline
    %B \pt\ & -2.4 (-7.3)  & +0.5 (+1.4) & +2.1 (+6.2) \\
    %$N^{close}_{trks}$ &  +2.8 (+8.4) & -0.7 (-2.0) & -2.0 (-6.1) \\
    %DOCA$_{xtrk}$ & -1.1 (-3.2) & +0.1 (+0.2) & +1.1 (+3.3) \\
    %B isolation $I_{0.7}$ & +1.2 (+3.5) & -0.0 (-0.1) & -1.0 (-3.0) \\
    %$|d_{0}|^{min}$ sig. & -0.8 (-2.3) & -0.1 (-0.4) & +1.0 (+2.9) \\
    %$\chi^2_{\mu,xPV}$ & +1.0 (+2.9) & +0.0 (+0.1) & -0.9 (-2.7) \\
    %DOCA$_{\mu\mu}$ & +0.7  (+2.1) & +0.1 (+0.3) & -0.7 (-2.1) \\
    %$|d_{0}|^{max}$ sig. & +0.9 (+2.6) & -0.2 (-0.7) & -0.5 (-1.6) \\
    %$\Delta{R}$ & -0.3 (-0.8) & -0.2 (-0.5) & +0.5 (+1.6) \\
    %$|\alpha_{2D}|$ & -0.3 (-0.9) & -0.1 (-0.4) & +0.5 (+1.5) \\
    %$L_{xy}$ &  +0.2 (+0.6) & -0.5 (-1.4) & +0.4 (+1.1) \\
    %$P_{L}^{min}$ & +0.2 (+0.7) & +0.2 (+0.7) & -0.4 (-1.1) \\
    %IP$_B^{3D}$ & +0.5 (+1.5) & -0.1 (-0.2) & -0.3 (-1.0) \\
    %$\log[\chi^{2}_{\mathrm{PV,SV}}]_{xy}$ & +0.6 (+1.8) & -0.2 (-0.6) & -0.3 (-0.9) \\
    %$\Delta\phi(\mu\mu)$ & -0.4 (-1.1) & -0.1 (-0.4) & -0.1 (-0.3) \\
    %\hline
    %quadratic sum & 4.0 (12.0) & 2.3 (6.9) & 4.5 (13.4) \\
    %\hline
    %linear sum   & +2.8 (+8.4)  & -1.3 (-3.9)  & -0.6 (-1.8) \\
    %\hline
  %\end{tabular}
    %\caption{Variation of the continuum-BDT efficiency in the three bins used
      %for extracting the \Bs\ signal. For each discriminating variable,
      %the ratio of the distribution in  \BsJpsiPhi\ data and simulation
      %is used to reweight the MC distribution for \Bs.
      %The table shows absolute and relative variation in the nominal efficiency of each BDT bin.}
%\label{tab:systBDTbin-BsJpsiphi}
%\end{table}

%\begin{figure}[!htb]
%\begin{center}
  %\hspace{-0.5cm}
  %\includegraphics[width=0.52\textwidth]{eps/contBDT/Bplus_contBDT-reweighed_jpsiK.eps}
  %\hspace{-0.5cm}
  %\includegraphics[width=0.52\textwidth]{eps/contBDT/Bplus_contBDT-reweighed_jpsiphi.eps}
%\caption{Continuum-BDT distributions observed for \BpKpJpsi\ and \BsJpsiPhi\ candidates,
  %compared to the corresponding ones from MC. 
  %In both cases, a linear fit is drawn to describe the ratio of data over MC. }
%\label{fig:systBDTbins}
%\end{center}
%\end{figure}


%\renewcommand\arraystretch{1.1}
%\begin{table}[!htb]
  %\centering
  %\begin{tabular}{l|c|c|c}
    %\hline
    %& \multicolumn{3}{c}{ absolute (relative) variation w.r.t. 33.3\% [\%] }\\
    %channel & bin 1 & bin 2 & bin 3 \\
    %\hline
    %\BpKpJpsi & +3.7 (+11.0)  & -0.1 (-0.3) & -3.6 (-10.9) \\
    %\BsJpsiPhi &  +3.0 (+9.1) & -0.1 (-0.3) & -2.9 (-8.7) \\
    %\hline
  %\end{tabular}
    %\caption{Variation of the continuum-BDT efficiency in the three bins
      %after reweighting the BDT distribution according to the data-to-MC
      %ratio observed in the \BpKpJpsi\ and \BsJpsiPhi\ channels.} 
%\label{tab:systBDTbin-BDTclassifier}
%\end{table}


%%Weights from JpsiPhi\\
%%classify the variables reweighting as a function of their impact in the last c-BDT bin
%%we can take this into account in the fit with Gaussian constraints on the fraction of signal
%%events in each bin (nominal value at 33.3\%



%%signal efficiency in each BDT bin:\\     This is J/psi phi with late BDT
%%bin 1 = 0.3635\\
%%bin 2 = 0.3323\\
%%bin 3 = 0.3042\\



%%Configuration of toy experiments:\\
%%generate toys with these fractions in each bin\\
%%fit assuming 1/3 of signal events in each bin


%\clearpage

\clearpage
